% -*- TeX:UTF-8 -*-
%%
%% KAIST 학위논문양식 LaTeX용 (ver 0.4) 예시
%%
%% @version 0.4
%% @author  채승병 Chae,Seungbyung (mailto:chess@kaist.ac.kr)
%% @date    2004. 11. 12.
%%
%% @requirement
%% teTeX, fpTeX, teTeX 등의 LaTeX2e 배포판
%% + 은광희 님의 HLaTeX 0.991 이상 버젼 또는 홍석호 님의 HPACK 1.0
%% : 설치에 대한 자세한 정보는 http://www.ktug.or.kr을 참조바랍니다.
%%
%% @note
%% 기존에 널리 쓰여오던 차재춘 님의 학위논문양식 클래스 파일의 형식을
%% 따르지 않고 전면적으로 다시 작성하였습니다. 논문 정보 입력부분에서
%% 과거 양식과 다른 부분이 많으니 아래 예시에 맞춰 바꿔주십시오.
%%
%%
%% @acknowledgement
%% 본 예시 논문은 물리학과 박사과정 김용현 님의 호의로 제공되었습니다.
%%
%% -------------------------------------------------------------------
%% @information
%% 이 예제 파일은 hangul-ucs를 사용합니다. UTF-8 입력 인코딩으로
%% 작성되었습니다. hlatex의 hfont는 이용하지 않습니다. --2006/02/11

% @class kaist.cls
% @options [default: doctor, korean, final]
% - doctor: 박사과정 | master : 석사과정
% - korean: 한글논문 | english: 영문논문
% - final : 최종판   | draft  : 시험판
% - pdfdoc : 선택하지 않으면 북마크와 colorlink를 만들지 않습니다.

%% 마지막엔 final로 해서 표지까지 다 나오도록
\documentclass[master,english,final]{kaist-ucs}
% \documentclass[master,english,draft]{kaist-ucs}


% If you want make pdf document (include bookmark, colorlink)
%\documentclass[doctor,english,final,pdfdoc]{kaist-ucs}

% kaist.cls 에서는 기본으로 dhucs, ifpdf, graphicx 패키지가 로드됩니다.
% 추가로 필요한 패키지가 있다면 주석을 풀고 적어넣으십시오,
%\usepackage{...}
\usepackage{amsfonts}
\usepackage{bm}
\usepackage{multirow}
\usepackage{amsmath}
\usepackage{algorithm}
\usepackage{algpseudocode}

\makeatletter
\renewcommand{\Function}[2]{%
  \csname ALG@cmd@\ALG@L @Function\endcsname{#1}{#2}%
  \def\jayden@currentfunction{#1}%
}
\newcommand{\funclabel}[1]{%
  \@bsphack
  \protected@write\@auxout{}{%
    \string\newlabel{#1}{{\jayden@currentfunction}{\thepage}}%
  }%
  \@esphack
}
\makeatother

% @command title 논문 제목(title of thesis)
% @options [default: (none)]
% - korean: 한글제목(korean title) | english: 영문제목(english title)
\title[korean] {상품 평가 품질 조작에 견고한 추천 시스템}
\title[english]{A Robust Recommendation System \linebreak Against Review Quality Manipulation}

% @note 표지에 출력되는 제목을 강제로 줄바꿈하려면 \linebreak 을 삽입.
%       \\ 나 \newline 등을 사용하면 안됩니다. (아래는 예시)
%
%\title[korean]{탄소 나노튜브의 물리적 특성에 대한\linebreak 이론 연구}
%\title[english]{Theoretical study on physical properties of\linebreak
%                carbon nanotubes}
%
% If you want to begin a new line in cover, use \linebreak .
% See examples above.
%


% @command author 저자 이름
% @param   family_name, given_name 성, 이름을 구분해서 입력
% @options [default: (none)]
% - korean: 한글이름 | chinese: 한문이름 | english: 영문이름
% 한문 이름이 없다면 빈 칸으로 두셔도 됩니다.
%
%
% If you are a foreigner , write your name in korean or your korean name.
% If you can't write native character, you can make the chinese blank empty
% Write as follow
% \author[korean]{family name in korean}{given name in korean}
% \author[chinese]{family name in your native language}{given name in your native language}
% \author[english]{family name in english}{given name in english}
%
\author[korean] {심}{동 진}
\author[korean2] {심}{동진}    %이름을 붙여 써 주시기 바랍니다.
\author[chinese]{沈}{東 鎭}
\author[english]{Sim}{Dongjin}

% @command advisor 지도교수 이름 (복수가능)
% @usage   \advisor[options]{...한글이름...}{...영문이름...}{signed|nosign}
% @options [default: major]
% - major: 주 지도교수  | coopr: 공동 지도교수
\advisor[major]{이 윤 준}{Yoon-Joon Lee}{signed}
\advisor[major2]{이윤준}{Yoon-Joon Lee}{signed}    %한글 성과 한글 이름을 모두 붙여 써 주시기 바랍니다.
\advisorinfo{Professor of School of Computing} %제출승인서에 들어가는 교수님 정보, advisor's information
%\advisor[coopr]{홍 길 동}{Gil-Dong Hong}{nosign}
%\advisor[coopr2]{홍길동}{Gil-Dong Hong}{nosign}    %한글 성과 한글 이름을 모두 붙여 써 주시기 바랍니다.
%
% 지도교수 한글이름은 입력하지 않아도 됩니다.
% You may not input advisor's korean name
% like this \advisor[major]{}{Chang, Kee Joo}{signed}
%


% @command department {학과이름}{학위종류} - 아래 표에 따라 코드를 입력
% @command department {department code}{degree field}
%
% department code table
%
% PH        // 물리학과 Department of Physics
% MAS   // 수리과학과 Department of Mathematical Sciences
% CH    // 화학과 Department of Chemistry
% NST   // 나노과학기술대학원 Graduate School of Nanoscience & Technology
% NT        // 나노과학기술 학제전공 Nano Science and Technology Program
% BS    // 생명과학과 Department of Biological Sciences
% BIS   // 바이오및뇌공학과 Department of Bio and Brain Engineering
% MSE   // 의과학대학원 Graduate School of Medical Science and Engineering
% BM    // 의과학 학제전공 Biomedical Science and Engineering Program
% CE    // 건설및환경공학과 Department of Civil and Environmental Engineering
% ME    // 기계공학전공 Division of Mechanical Engineering
% AE    // 항공우주공학전공 Division of Aerospace Engineering
% OSE   // 해양시스템공학전공 Department of Ocean Systems Engineering
% CBE   // 생명화학공학과 Department of Chemical and Biomolecular Engineering
% MS    // 신소재공학과 Department of Materials Science and Engineering
% NQE   // 원자력 및 양자공학과 Department of Nuclear and Quantum Engineering
% EEW   // EEWS 대학원 Graduate School of EEWS
% PSE   // 고분자 학제전공 Polymer Science and Engineering Program
% SPE   // 우주탐사공학 학제전공 Space Exploration Engineering Program
% ENY   // 환경・에너지공학 학제전공 Environmental and Energy Technology Program
% MSB   // 경영과학과 Department of Management Science
% IT        // 경영과학과(IT경영학) Department of Management Science (IT Business)
% BAP   // 경영전문대학원프로그램 Master of Business Administration Program
% ITP   // 글로벌IT기술대학원프로그램 Global Information & Telecommunication Technology Program
% ITM   // 기술경영전문대학원 Graduate School of Innovation & Technology Management
% GCT   // 문화기술대학원 Graduate School of Culture Technology
% CT    // 문화기술(CT) 학제전공 Culture Technology Program
% EE    // 전기및전자공학부 Department of Electrical Engineering
% CS    // 전산학과 Department of Computer Science
% ICE   // 정보통신공학과 Department of Information and Communications Engineering
% IE        // 산업 및 시스템 공학과 Department of Industrial & Systems Engineering
% KSE   // 지식서비스공학과 Department of Knowledge Service Engineering
% ID        // 산업디자인학과 Department of Industrial Design
% RE    // 로봇공학 학제전공 Robotics Program
% STE   // 반도체 학제전공 Semiconductor Technology Educational Program
% SEP   // 소프트웨어공학프로그램 Software Engineering Program
% TE    // 정보통신공학 학제전공 Telecommunication Engineering Program
% MT    // 경영공학과 Department of Management Engineering
% TM    // 테크노경영전공 Techno-MBA Program
% FIN   // 금융전문대학원 Graduate School of Finance and Accounting
% DMP   // 디지털미디어프로그램 // Digital Media Program
% IMBA // IMBA // IMBA
% EM // 이그제큐티브MBA // Executive MBA
% SM // 사회적기업가MBA // Social Enterpreneurship MBA
% PM // 프로페셔널MBA // Professional MBA
% FEM // 금융이그제큐티브MBA // Financial Executive MBA
% GM // 녹색MBA // Green MBA
% GBP // 녹색경영정책프로그램 // Green Business and Policy Program
% IMM // 정보미디어MBA // Information and Media MBA
%
% science: 이학 | engineering: 공학 | business : 경영학
% 박사논문의 경우는 학위종류를 입력하지 않아도 됩니다.
% If you write Ph.D. dissertation, you cannot input degree field.
% The third parameter : a | b | c
% a: 소속된 학과만 쓰는 옵션 (학과에만 소속되어 있는 경우에는 무조건 a를 선택해야 함)
% b: 학과 아래의, 프로그램이나 학제전공에 소속되어 있을 경우에 학과와 프로그램을 함께 쓰는 옵션
% c: 학과 아래의, 프로그램이나 학제전공에 소속되어 있을 경우에 학과를 쓰지 않고 프로그램이나 학제전공의 이름만 쓰는 옵션
%
% a: it represents only the name of department. (if you aren't in the program under the department, must choose a)
% b: it represents the names of department and the program that is under the department (consider this when you are in the program not only department)
% c: it represents only the name of program that is under the department (consider this when you are in the program not only department)
\department{CS}{engineering}{a}

% @command studentid 학번(ID)
\studentid{20153338}

% @command referee 심사위원 (석사과정 3인, 박사과정 5인)
\referee[1]{이 윤 준}
\referee[2]{김 명 호}
\referee[3]{유 신}
%\referee[4]{가 동 호}
%\referee[5]{박 태 현}
% \referee[5] {Barack Obama}
% Of course english name is available

% @command approvaldate 지도교수논문승인일
% @param   year,month,day 연,월,일 순으로 입력
\approvaldate{December}{19}{2016}

% @command refereedate 심사위원논문심사일
% @param   year,month,day 연,월,일 순으로 입력
\refereedate{2016}{12}{19}

% @command gradyear 졸업년도
\gradyear{2017}

% 본문 시작
\begin{document}

    % 앞표지, 속표지, 학위논문 제출승인서, 학위논문 심사완료 검인서는
    % 클래스 옵션을 final로 지정해주면 자동으로 생성되며,
    % 반대로 옵션을 draft로 지정해주면 생성되지 않습니다.

    % 논문 서지, 초록, 핵심 낱말, 영문 초록, 영어 핵심 낱말 (Information of thesis, abstract in korean, keywords in korean, abstract in english, keywords in english)
   \thesisinfo
   %% Letters of abstract in korean must be less than 500 and words of abstract in english must be less than 300.
   %% Number of keywords must be less than 6.
   %% Don't write english letters in the abstract in korean.
    \begin{summary}
    추천 시스템은 사용자들의 구매 결정에 주요한 영향을 주기 때문에 악의적인 조작의 표적이 되어왔다.
    대표적인 추천 방법은 상품에 대한 평가들이 정직하다는 가정하에 사용자의 평점을 분석하는 협업 필터링 방법이다.
    하지만, 협업 필터링 방법은 거짓된 평점을 주입하는 실링 공격에 의해 조작될 위험이 있다.
    이로 인해 실링 공격의 영향력을 낮추는 여러 방법이 제안되었다.
    한편 많은 추천 시스템은 사용자들에게 다른 사용자의 리뷰 유용성을 평가하길 권장한다.
    이러한 과정을 통해 거짓된 리뷰는 사용자들에 의해 분류될 수 있을 거라는 가정하에 연구자들은 리뷰의 유용성을 고려한 추천을 통해 실링 공격을 극복하고자 하였다.
    하지만 이러한 추천 방법은 거짓된 리뷰의 유용성 평가 데이터가 주입된다면 오히려 조작의 정도가 심해질 위험이 있다.
    본 논문에선 리뷰의 거짓 평점 데이터의 영향력을 낮추어 진실한 유용성을 추정함으로써 더욱 견고한 추천 방법을 제안한다.
    실제 데이터를 토대로 한 실험 결과를 통해 제안하는 방법의 견고함을 보였다.
    \end{summary}

    \begin{Korkeyword}
    견고한 추천 시스템, 협업 필터링, 추천 시스템 조작, 리뷰 유용성 측정
    \end{Korkeyword}


    \begin{abstract}
%     Nowadays, many customers refer to recommendation systems when buying items, and hence recommendation
% systems have become an attractive target of manipulation. Collaborative ltering, widely adopted
% in recommendation systems, exploits the observed item ratings of users to provide personalized recommendations
% under the assumption that all users rate items honestly. Unfortunately, however, injecting
% multiple fake item ratings i.e. shilling attack can easily manipulate recommendation systems with the
% naive assumption. There are several approaches to remove and mitigate the eect of shilling attack.
% Recently, some researchers focused on that most recommender systems encourage users to not only leave
% reviews about items but also rate the helpfulness of reviews written by other users based on review
% content. With the assumption that fake reviews are rated as not helpful reviews by users, they suggested
% a recommendation considering the helpfulness of reviews. However, their recommendation is vulnerable
% to injecting fake helpfulness ratings with intent to boost the helpfulness of fake reviews. In this paper,
% we propose a review helpfulness measure which is robust against manipulating review helpfulness attack
% to provide unbiased review helpfulness for the recommendation algorithm considering review helpfulness.
% In particular, we present a user similarity measure exploiting properties of injected users for the attack
% and a robust helpfulness measure mitigating the eect of suspicious helpfulness ratings. Experimental
% results on a real-world dataset consisting of item rating and helpfulness rating indicate the robustness
% of our method.
    Recommendation systems influence decision making, and hence have become attractive targets of manipulation.
    Collaborative filtering, widely adopted in recommendation systems, exploits the observed ratings given to items by users to provide personalized recommendations under the assumption that all users honestly rate items.
    Unfortunately, however, shilling attacks which inject multiple fake reviews can easily manipulate recommendation systems with the naive assumption.
    There are several approaches to mitigate the effect of shilling attack.
    Recently, some researchers have been interested in the fact that most recommendation systems encourage users to write reviews, as well as rate the usefulness of reviews written by other users based on their review content.
    With the assumption that fake reviews are rated as not helpful by users, they suggest recommendation systems considering the helpfulness of reviews.
    However, their recommendations are vulnerable to attacks that inject fake helpfulness ratings with intent to boost the helpfulness of fake reviews.
    In this paper, we propose a robust recommendation system to overcome such attacks.
    The proposed system estimates the true helpfulness of reviews, thereby preventing such attacks from manipulating recommendation results.
    Experimental results on a real-world dataset indicate the robustness of our method.
    %Therefore, we propose a new review helpfulness measure which estimates the true helpfulness of reviews to robustify the recommendation algorithm considering review helpfulness.
    % First, the measure used to estimate the true helpfulness of reviews is described.
    % we propose a new review helpfulness measure which estimates the true helpfulness of reviews to robustify recommendation systems against such attacks


    %In particular, we present a user similarity measure exploiting properties of attack and a robust helpfulness measure mitigating the effect of suspicious helpfulness ratings.
    %we map users into feature vectors which encode patterns associated with behavior
    %restricts suspicious helpfulnes ratings
    \end{abstract}

    \begin{Engkeyword}
    Robust recommendation system, Collaborative filtering, Shilling attack, Review helpfulness
    \end{Engkeyword}





    % 목차 (Table of Contents) 생성
    \tableofcontents

    % 표목차 (List of Tables) 생성
    \listoftables

    % 그림목차 (List of Figures) 생성
    \listoffigures

    % 위의 세 종류의 목차는 한꺼번에 다음 명령으로 생성할 수도 있습니다.
    %\makecontents

%% 이하의 본문은 LaTeX 표준 클래스 report 양식에 준하여 작성하시면 됩니다.
%% 하지만 part는 사용하지 못하도록 제거하였으므로, chapter가 문서 내의
%% 최상위 분류 단위가 됩니다.
%% You cannot use 'part'

\chapter{Introduction}

Since recommender systems influence purchase decisions, their positive recommendation can lead to significant monetary benefit for product sellers.
According to \cite{yelp_study} , increasing overall rating of business by one star on Yelp can increase its revenue by 9\%.
Unfortunately, however, the strong impact of recommender systems has attracted malicious attackers who try to bias recommendation result to increase overall rating of their target items.

Matrix factorization(MF) model based collaborative filtering(CF), one of the most common approaches for the recommendation, infers users’ tastes and items’ attributes based on the observed ratings of users for items and recommends products whose attributes match a user’s taste.
Standard MF assumes that all the observed ratings are conducted by honest users.
However, this assumption is easily violated in practice due to the presence of attackers.
Due to the open nature of recommendation systems, attackers can inject multiple fake users (Shillers) and fake ratings to increase overall rating of their target items on the recommendation system.
%[shilling attack].
Such injections with intent to bias recommendation are called, shilling attacks.
%[shilling attack strategies study]
%[ shilling attack detection]
%[robust recommendation to prevent shilling attack]
There are a number of studies about shilling attack strategies , shilling attack detection  and robust recommendation to prevent shilling attack .

Recent studies \cite{DualRole,RQMF} focused on the following dual roles of users in recommender systems.
In real-world recommender systems, users play as reviewers that write reviews about items in the form of a numeric rating score (such as a 1~5-star rating) accompanied by review text, and they also play as helpfulness raters that rate the helpfulness of reviews based on review content by giving numeric rating score.
Figure \ref{item_review_example} contains a review of \textit{Captain America: Civil War (DVD)} written by a user \textit{lmmyvasi29} with a 5-star rating.
Figure \ref{helpfulness_rating_example} represents how other users rate the helpfulness of the review in Figure \ref{item_review_example}.

Suhang et al. \cite{DualRole} treat helpfulness rating as users’ implicit feedback about items, and hence incorporates helpfulness rater role of users in recommendation systems to mitigate the data sparsity and cold-start problems.
Sindhu et al. \cite{RQMF} suggest recommendation taking into account review quality to improve the performance in the presence of spurious reviews.
They assume that if the number of helpfulness ratings is sufficiently high, spurious reviews get negative helpfulness ratings.
They measure the quality of a review by aggregating the helpfulness ratings about the review and mitigate the impact of low-quality reviews on optimizing parameters of recommendation model.
These studies show that incorporating review helpfulness information has potential benefits of improving the performance and robustness of recommendation systems.

However, these studies do not deal with manipulating review helpfulness attack.
After injecting fake reviews, malicious attackers can easily inject many fake helpfulness ratings to promote the helpfulness of fake reviews.
The helpfulness measure vulnerable to injected fake helpfulness ratings results in amplifying the negative effect of fake reviews rather than mitigating it.

Therefore, in this paper, we propose a robust recommendation in the presence of fake reviews and helpfulness rating via unbiased review helpfulness measure.
Our approach to a robust recommendation system consists of three stages.
The first stage involves the task of mapping users to a representation vector space such that users who are similar in terms of behaviors related to shilling attacks are located in close proximity to one another in the space.
In the second stage, the helpfulness of each review is measured by the weighted mean of the helpfulness ratings associated with each review.
A helpfulness rating is “down-weighted” if the similarity between the representation vectors of the helpfulness rater and the writer of the associated review is above the predetermined threshold.
In the final stage, the helpfulness of each review measured in the previous stage is used to collaborative filtering.
We adopt the cost function suggested by Sindhu et al. \cite{RQMF}.
We demonstrate the effectiveness of the proposed method by measuring the effect of various attacks on a real dataset.
Compared with other methods, proposed method mitigates the effect of manipulating review helpfulness.

The rest of paper is organized as follows.
Chapter 2 presents background of this paper.
Related work are described in Chapter 3.
In Chapter 4, we formally define our attack model.
Chapter 5 presents the proposed method.
In Chapter 6 presents the experimental methodology used to evaluate the robustness of our approach and results of our experiment.
Finally, we conclude in Chapter 7.


%%
%% 그림 삽입 예시
%% Example. how to insert graph
%%
%% Note. 가급적 \includegraphics 명령을 사용하십시오.
%% Recommen : Use \includegraphics to insert graph.
%%
\begin{figure}[h]
    \centerline{\includegraphics[width=7.5cm]{figure/item_review_example}}
    \caption{ An item review example    } \label{item_review_example}
\end{figure}

\begin{figure}[h]
    \centerline{\includegraphics[width=5.5cm]{figure/helpfulness_rating_example}}
    \caption{ A helpfulness rating example    } \label{helpfulness_rating_example}
\end{figure}

\chapter{Background}
\section{Notation}

Throughout this paper, sets are denoted as italic capital letters.
Let $U = \{u_1,u_2,…,u_n\}$ and $I = \{i_1,i_2,…,i_m\}$  be a set of users and items where $n$ and $m$ are the number of users and items, respectively.
In recommendation systems, users can rate items in the form of a numeric rating score accompanied by review text.
Unless otherwise mentioned, the term ‘review’ of a user for an item indicates numeric rating the user give to the item.
We use the matrix $ \bm{R} \in \mathbb{R}^{n \times m} $ to denote the user-item rating matrix where an entry $ R_{i,j} $ indicates the rating score of the $i$-th user for the $j$-th item.
Note that the rating matrix $\bm{R}$ is sparse since users usually rate a small set of items.
If the $i$-th user did not rate the $j$-th item, then we assign “?” to the missing rating $\bm{R}_{i,j}$.
We use $IR=\{(u,i,ir)| u_a \in U,i_b \in I,r=\bm{R}_{a,b},r \neq ? \}$ to represent item rating dataset where $(u,i,ir)$ means user $u$ rates item $i$ with score $ir$.

Some recommendation systems allow users to rate the helpfulness of other users’ reviews with intent to improve the user experience.
In such systems, after reading review text of another user, users give the review helpfulness score in the form of a numeric rating score.
For example, $(u_a,u_b,i_c,hr)$ represents user $u_a$ gives helpfulness score $hr$ to the review of user $u_b$ for item $i_c$.
We use the tensor $\bm{H} \in \mathbb{R}^{n \times n \times m}$ to denote the user-user-item helpfulness rating tensor where an entry $\bm{H}_{a,b,c}$ indicates the review helpfulness score that the $a$-th user gives to the review given by the $b$-th user about the $c$-th item.
Similarly with item rating dataset $IR, HR=\{(u_a,u_b,i_c,h)| u_a,u_b \in U,i_c \in I,hr=\bm{H}_{a,b,c},hr \neq ?\}$ represents helpfulness rating dataset.

Since we consider an attacker who injects fake users and fake ratings, we use $U^g$ and $U^f$ to denote the set of genuine users and fake users, respectively.
Genuine item rating and helpfulness rating dataset are denoted by $IR^g=\{(u,i,r)|u \in U^g\}$ and $HR^g=\{(u,v,i,h)|u \in U^g\}$, respectively.
Similarly, fake item rating and helpfulness rating dataset are denoted by $IR^f=\{(u,i,r)|u \in U^f\}$ and $HR^f=\{(u,v,i,h)|u \in U^f\}$, respectively.
%the maximum rating on the rating scale.
Unless otherwise noted, the range of item rating score is from 1 to 5, and helpfulness rating score ranges from 0 to 5.




\section{Matrix Factorization Based Collaborative Filtering}
In the context of collaborative filtering, matrix factorization infers latent features of user and items based on observed ratings of users for items.
It decomposes a user-item rating matrix R into two latent matrices $\bm{U} \in \mathbb{R}^{n \times d}$ and $\bm{V} \in \mathbb{R}^{d \times m}$ corresponding latent features of user and item, respectively where the $d$ is the number of latent features.
In specific, given an observed rating matrix $\bm{R}$, matrix factorization optimizes two matrices $\bm{U}$ and $\bm{V}$ by minimizing the following cost function which is the sum of prediction error terms and regularization terms.

\begin{equation}
Cost(\bm{U},\bm{V} | \bm{R})=\sum_{\bm{R}_{i,j} \neq ?} (  \bm{R}_{i,j} - (\bm{UV})_{i,j} )^2 + \lambda(||\bm{U}||_F^2+||\bm{V}||_F^2)
\end{equation}
After obtaining the optimized U and V, the missing ratings in the rating matrix R are predicted via the dense matrix which is the product of U and V.
The figure \ref{mf_base} indicates a toy example.
Observed rating matrix (Figure \ref{mf_base} (a)) represents user-movie rating matrix.
The item set of the rating matrix consists of two romance movies, two horror movies, and one bad movie.
The user set of the rating matrix consists of two romance movie lovers and two horror movie lovers.
By applying matrix factorization on the rating matrix R, prediction matrix captures the tastes of users and judges that users will not like the 5-th movie (bad movie).


\begin{figure}[h]
    \centerline{\includegraphics[width=12.5cm]{figure/mf_base}}
    \caption{ Matrix factorization example    } \label{mf_base}
\end{figure}


\section{Shilling Attack}

Shilling attacks generate fake users (shillers) which give fake ratings to particular items with the intent to bias recommendation results of recommendation system.
Shilling attacks are categorized into two categories: push attacks inject fake users which give high ratings to particular items to promote the recommendation score for the particular items, while nuke attacks inject fake users which give low ratings to particular items aiming at decreasing the popularity of the items.
The goal of Shilling attacks is to manipulate the recommendation results for the particular items for other normal users.
However, injecting fake rating associated with target items only is not enough to manipulate collaborative filtering based recommendation systems.
Due to the principle of collaborative filtering, collaborative filtering recommends normal users to the direction shilling attacks wish if it recognizes that normal users are similar to fake users.
In order to exploit the principle of collaborative filtering fully, fake users of shilling attack have to mimic rating behaviors of normal users.
There are various attack models about how to mimic rating behavior of normal users: Random attack, Average attack, Bandwagon attack.
%[Burke et al. 2006].
In the context of push attack, Random attack injects fake users who give the highest rating to their target items and rate the randomly chosen items around the overall mean.
Average attack generates fake user that give the highest rating to their target items and the mean rating of each item to randomly chosen items.
Bandwagon attack consists of fake users whose ratings for their target items and popular items are maximum.

The figure \ref{mf_attacked} shows an example of shilling attack and its effect. We inject two shillers whose aims are boosting the prediction score of the bad movie.
Shillers rate the bad movie with the highest possible rating value, i.e. 5, and rate other movies in a similar way to other genuine users.
Matrix factorization lowers the error of fake ratings and therefore misjudges the prediction scores of the bad movie.
Compared with figure \ref{mf_base}, figure \ref{mf_attacked}contains high predicted rating of genuine users for the bad movie.

\begin{figure}[h]
    \centerline{\includegraphics[width=12.5cm]{figure/mf_attacked}}
    \caption{  Matrix factorization in the presence of shilling attack   } \label{mf_attacked}
\end{figure}

\chapter{Related Work}
Our paper is related to the following topics: robust recommendation and review helpfulness. We discuss the two topic in the following subsections.

\section{Robust Recommendation}
% robust statistical model
Bhaskar et al. \cite{RMF} propose Robust Matrix Factorization (RMF) using M-estimators to bound the effect of outliers and noisy data. \cite{LiesAndPropaganda,UnsupervisedShilling,AttackResistant} apply PCA-based variable selection to detect suspicious users in unsupervised setting.


%Social Network
%[Exploring the design space of social network-based Sybil defenses의 ref 2~11]
There are studies  leveraging social networks to defend against Shilling attacks.
%Exploring the design space of social network-based Sybil defenses]
Corresponding to [], these studies try to label users in the recommendation system either ‘untrustworthy’ or ‘trustworthy’, or bound the advantage an attacker can gain by injecting fake users. The common assumption of these studies is that a fake user cannot establish an arbitrary number of links to genuine users.

% Fake review detection - (supervised) textual feature, meta-feature
%[Detecting Product Review Spammers using Rating Behaviors CIKM 10]
%리뷰에 담긴 feature를 분석해서 좋은 글인지 아닌지 구분하거나 글자 길이 리뷰어의 평판 정도 등을 feature로 하여 review의 genuinity를 classification하여 attack으로 분류된 리뷰는 제외하고 CF를 돌리는 방법을 택했다.
Many researchers take supervised approaches to detect fake reviews based on the textual or meta features in review.
However, supervised learning based fake review detections have several drawbacks.
In general, High class imbalance in the number of genuine data and attack label data degrades the performance of supervised learning.
Furthermore, Review text-based approach is not proper to be adopted in general due to its domain-specific property.

\section{Review Helpfulness}
Many E-commerce sites encourage users to rate the helpfulness of reviews and place helpful reviews in a position where many users access to improve the overall user experience.
Users can examine the helpfulness of reviews with statistics such as “90 (out of 100) people found this review helpful” or “40 members have rated this review on average (somewhat helpful)”.
%[++ DualRole, ETAF]

Motivated by this, some researchers \cite{RQMF} propose a collaborative filtering method incorporating the helpfulness of reviews as the weight of reviews in the training phase.
Under the assumption that spam reviews receive bad helpfulness ratings, they lower the weight of unhelpful reviews prediction, thereby reducing the contribution of unhelpful reviews to missing rating prediction.
Kim et al. \cite{naive_helpfulness} propose the measure below to quantify helpfulness for an item review by aggregating helpfulness ratings for the item review.
\begin{equation}
Helpfulness(review (u,i))=\frac{1} {N_{\bm{H}_{v,u,i}}} \sum_{\bm{H}_{v,u,i} \neq ?} \bm{H}_{v,u,i}
\end{equation}
where $N_{\bm{H}_{v,u,i}}$ is the number of helpfulness ratings for item rating $\bm{R}_{u,i}$.

They use the following cost function proposed by \cite{ImplicitCF} to treat the helpfulness of a review as the weight of the review.
\begin{equation}
Cost(\bm{U},\bm{V} | \bm{W} \bm{R})=\sum_{\bm{R}_{i,j} \neq ?} \bm{W}_{i,j}(  \bm{R}_{i,j} - (\bm{UV})_{i,j} )^2 + \lambda(||\bm{U}||_F^2+||\bm{V}||_F^2)
\end{equation}
In this cost function, prediction error term changes from sum of squared errors to weighted sum of squared errors.
In other words, this cost function measures the error of a weighted matrix factorization.
Therefore, the weighted sum of errors allows unhelpful reviews to have a significant error but penalizes error of helpful reviews heavily.
They demonstrate considering review helpfulness could improve the overall performance of recommendation in the presence of spam review.
Figure \ref{wmf_good} shows an example where the weights of genuine reviews are bigger than that of fake reviews.
With well-assigned weight matrix, latent features of users and items be able to explain genuine review, and hence shillers fail to manipulate prediction of genuine users for the bad movie.

\begin{figure}[t]
    \centerline{\includegraphics[width=12.5cm]{figure/wmf_good}}
    \caption{  Weighted matrix factorization with well assigned weight matrix  } \label{wmf_good}
\end{figure}

\begin{figure}[t]
    \centerline{\includegraphics[width=12.5cm]{figure/wmf_bad}}
    \caption{  Weighted matrix factorization with badly assigned weight matrix  } \label{wmf_bad}
\end{figure}

The review helpfulness measure relies on the naive assumption which trusts all helpfulness rating.
However, this assumption is easily violated if an attacker injects fake helpfulness ratings to promote the helpfulness of their fake reviews.
From an adversarial perspective, the cost of fake helpfulness rating injection is not much more expensive than the cost of fake item rating injection.
Figure \ref{wmf_bad} shows the case an attacker successes to manipulate the helpfulness of fake reviews.
The fake ratings for the bad movie have a larger weight than other normal users' ratings as shown in Figure \ref{wmf_bad}.
Due to the manipulated weight matrix, collaborative filtering outputs prediction matrix weighted toward fake reviews.

\chapter{Problem Definition}

\section{Attack Model}
Since push attack more directly promotes monetary benefit than nuke attack does, this paper only focuses on push attack. We leave nuke attack case for future work. The objective of our attack model is to increase overall users’ predicted ratings of target items.
% // Target items are poorly rated by some genuine users.
Our attack model involves in injecting fake item rating and helpfulness rating dataset, $IR^f$ and $HR^f$.
\subsection{Fake Item Rating Injection}
Since Random attack is known to be not effective, this paper focuses on Average attack and Bandwagon attack for injecting fake item rating dataset.
%[KDD 06]
Similarity with [], from the view of each fake user, the item set $I$ is partitioned into 4 groups, $I^target,I^{popular},I^{filler}$  and $I^{none}$ where $I^{target}$ is a set of target items; $I^{popular}$ is a set of popular items; $I^{filler}$ is a set of randomly chosen items which are referred to filler items; and $I^{none}$ is the set of unrated items, i.e. $I^{none}=I-I^{target}-I^{popular}-I^{filler}$. Each fake user rates items of $I^{target},I^{popular}$ and $I^{filler}$ as follows.

First, each fake user rates all target items with the highest possible rating score.  Secondly, with the aim of being similar to other users, each fake user gives rating to each filler item of $I^{filler}$ and its item rating is the average of all item rating given to the filler item. Finally, to increase the probability of being similar to a large number of other users, each fake users give the maximum item ratings for all popular items of $I^{popular}$. In summary, general form of fake item rating dataset $IR^f$ is defined as follows.
\begin{equation}
{IR}^f = \bigcup_{u^f \in U^f} \{(u^f,i^t,ir_{max}) | i^t \in I^{target} \} \cup \{(u^f,i^f,ir_{avg} {(i^f)}) | i^f \in I^{filler} \} \cup \{(u^f,i^p,ir_{max}) | i^p \in I^{popular} \}
\end{equation}
$ir_{max}$  is the maximum rating on the item rating scale and $ir_{avg} {(i^f)}$ is the average item rating of item $i^f$.
%The size of each group varies, however, since attack associated with the too big size of I_target, I_popular, I_filler can be easily detected as outlier; we restrict each size to less than 1% of the size of entire item set I.
%Attackers usually realize shilling attacks by injecting an attack profile as shown in Fig. 1, which is first defined by Bhaumik et al. (2006), Mobasher et al. (2007a) to mislead the CF system. Such profiles can be defined as four set of items (Bhaumik et al. 2006; Mobasher et al. 2007a). [from Shilling attacks against recommender systems: a comprehensive survey]

\subsection{Fake Helpfulness Rating Injection}
An attack whose goal is promoting the helpfulness of fake reviews involves in fake helpfulness rating injection. Such injection makes our study differ from existing attack models. Each fake user gives the highest helpfulness rating value to all the fake reviews about target items. Additionally, each fake user generates random helpfulness ratings for normal reviews to avoid being outlier. We call such helpfulness ratings as camouflage helpfulness ratings.
%/*We restrict the size of camouflage helpfulness ratings a fake user makes to less than the size of the helpfulness ratings the fake users makes. */
Formally, fake helpfulness rating dataset $HR^f$ is defined as follows:
\begin{equation}
HR^f = \bigcup_{u^f \in U^f} \{(u^f,v^f, i^t,hr_{max}) | v^f \in U^f,i^t \in I^{target} \} \cup \{(u^f,v,i,hr_{random}) | v \in U^g, i \in I, \bm{R}_{v,i} \neq ? \}
\end{equation}

$hr_{max}$  is the maximum rating on the helpfulness rating scale and $hr_{random}$ is a random rating.
In the presence of such a fake helpfulness rating dataset, computing the average helpfulness rating incurs a biased measurement and therefore amplifies the negative effect of fake reviews.

\section{Problem Definition}
We define our problem as: given an item rating and review helpfulness rating dataset in the presence of our attack model, estimate the true review helpfulness which is incorporated into the weighted matrix factorization based CF so that predictions for the missing ratings are insensitive to injected fake data.


\chapter{Proposed Method}
This section describes our method in detail.
The following sections describe how to capture suspicious helpfulness ratings and how to estimate true quality of reviews.
% We define the quality measure of a review as a function F of the helpfulness ratings for the review.
% \cite{RQMF} defines the function F as the average, which might lead to biased review quality.
% Considering fake helpfulness ratings, we define the function F as the Bayesian weighted mean.
% Then, we need to determine the weight of each helpfulness rating.
% Our goal is to assign low weight to suspicious helpfulness rating.
% \begin{equation}
% quality\(review r\) = F(HR)
% \end{equation}
% We suspect a helpfulness rating, if the users involved in the helpfulness rating are too similar.
% % ver 0.9
% %++ (review helpfulness가 recommendation 할 때 어떻게 적용되는가).
% Since the naive review helpfulness measure which takes the average of helpfulness ratings is easily manipulated by fake helpfulness ratings injection, fake reviews with biased helpfulness have more impact on a recommendation considering review helpfulness.
% Therefore, we propose a robust review quality measure to prevent fake helpfulness ratings from disturbing in evaluating the true quality of reviews so that a review helpfulness aware recommendation is resistant to fake reviews.
% In specific, we lower the weight of suspicious helpfulness ratings and measure the helpfulness of a review by a weighted average of helpfulness ratings for the review.
% The key challenge of this strategy is to measure how suspicious a given helpfulness rating is.
% The following sections describe how to capture suspicious helpfulness ratings and how to estimate true helpfulness of reviews.
% RQMF
\section{User2Vec}
% feature learning !!!
% Word2Vec takes a text corpus as its training data and learns a mapping of words to a vector space such that words that frequently appear together in sentences have similar vectors.
Recently, various prediction tasks [REF] have improved performance by learning the desirable features themselves, instead of manually determining domain-specific features.
Skip-gram model \cite{Word2Vec} is a popular model proposed for natural language processing task by Mikolov et al.
The goal of the Skip-gram model is to capture semantic relationships between words.
With the hypothesis that words which frequently appear together in sentences have semantic relationships, the Skip-gram model takes large real-world text corpus as training data and learns feature representations for words.
In specific, it map words to a feature space such that words frequently appear together in sentences have similar feature vectors.
The Skip-gram model is widely adopted for natural language processing task due to the efficiency and ability to capture useful relationships in the text data.
% example ) feature learning for network
Inspried by the success of the Skip-gram model, some researchers apply the Skip-gram model to learning a mapping of vertices of a network to vectors which encode social relation \cite{DeepWalk,Node2Vec}.
With the assumption random walk traces contain social relation between vertices, they generate samples of random walk traces as sequences of vertices and feed them into the Skip-gram model.

% analogy between
In this paper, we propose User2Vec, an algorithm for learning feature representations for users in recommendation system to detect such suspicious relationships between users.
\cite{Word2Vec} uses real-world sentences as sequence of semantically related words \cite{DeepWalk,Node2Vec} generates random walk traces as sequence of socially related vertices to obtain useful features for various prediction tasks.
Similarty to such approaches, we generate sequences of attack-related users and feed them into the Skip-gram model to obtain feature representations of user which is useful to detect fake users.

To generate sequences of attack-related users, we focus on behaviors of fake users.
%[Lies and Propaganda (IUI 07), Unsupervised shilling (AAAI 07)]
Several studies \cite{LiesAndPropaganda,UnsupervisedShilling} reported that fake users need to work together to maximize the effect of their attack.
This strategy is referred to as group attack.
In our attack model, fake users equally give the highest item rating to target items and the highest review helpfulness rating to their fake reviews ($(IR^{target},HR^{target})$).
Taking this into consdieration, we regard following relationships between users as clues to the group attack

\begin{enumerate}
\item Both user X and Y give $ir_{max}$ for an item
\item Both user X and Y give $hr_{max}$ for a review
\item User X gives $hr_{max}$ for a review written by User Y
\end{enumerate}

We refer to the users rate an item with $ir_{max}$ as enthusiasts for the item.
Similarly, we refer to the users rate the helpfulness of a review with $hr_{max}$ as supporters for the review.
The first (second) relationship represents the pair of enthusiasts (supporters) whose opinions about some item (review) are same.
If user $u_c$ and $u_d$ always rate in the same way items or reviews, it is reasonable to suspect that $u_c$ and $u_d$ are performing group attack.
The last relationship indicates the pair of a reviewer and a supporter.
If user $u_a$ always assigns the maximum helpfulness rating to all reviews written by another user $u_b$, then one can doubt that user $u_a$ intentionally promote the influence of user $u_b$.
Note that we only target the ratings with the highest score only since we focus on push attack.
Of course, pairs of normal users could reveal clues to the group attack due to the coincidence of opinions about items or reviews.
However, all fake users have to involve in many connections through the relationships associated with group attack as a necessity to maximize the degree of manipulation.
With this in mind, we suspect the truthfulness of a helpfulness rating if the rater and the reviewer are frequently connected through the mentioned relationships
% Hence, we hypothesize that the users created from group attacks will frequently be connected through the following relationships.

% In this paper, aimed at learning a mapping users to vector space such that fake users injected by group attack have similar vectors,
User2Vec consists of two steps.
In the first step, we sample user pairs that reveal clues to the group attack.
Sampling user pairs corresponding to the first relationship proceeds as follows.
Among the items having at least two reviewers who rate the item with the highest possible rating, we first sample an item with the probability proportional to the cardinality of the users associated with the item and choose two reviewers for the sampled item uniformly at random.
Sampling user pairs corresponding to the second and last relationship associated with the group attack involves in sampling reviews.
For a review to be a sample, it should receive at least two highest ratings.
The probability of sampling a review is proportional to the number of the helpfulness raters for the review.
We choose two helpfulness raters who rate the review with the highest possible rating for the sample related to the second relationship.
With a sampled review, we sample one supporter for the review and produce a pair of reviewer and supporter.
In the last step, we feed the sampled user pairs into the Skip-gram model and obtain feature vectors of users.
We expect the obtained feature vectors encode group attack patterns.
In other words, fake users are very closely located to each other in the feature space, while normal users are scattered.
Note that User2Vec, which places the fake users very close to each other in the feature space, does not guarantee that normal users are positioned away from each other in the feature space.
However, if the dimension of the feature space is moderately high, the probability that the similarity of two arbitrarily selected users is high is very small.
Therefore, although there is a risk of judging false positives, we judge that the relationship between two users with very high similarity is not trustful and define the suspiciousness of a helpfulness rating as a function of the similarity between the feature vectors of the helpfulness rater and the reviewer.


% editing!!!!!!!!!!!!
%+++ (이렇게 하는 이유와 discussion)
%The rationale behind using Word2vec model instead of counting the number of suspicious connection between users is that …[Don’t count, predict! A systematic comparison of context-counting vs. context-predicting semantic vectors]
%Word2vec를 이용하는 이유 그냥 count해서 user간의 connection을 그냥 count하지 않고 embedding을 하는 이유 – count로 하면 어느 정도가 의심스러운 지 정하기 모호하다. 또한 embedding을 보통 사용자처럼 보이기 위한 데이터를 넣었더라도 추천 시스템을 bias하기 위해 Fake user가 하는 group effect를 주목하는 것
%Though we focus on only item rating and helpfulness rating data in this paper, other additional information, like a social network, can be used to generate pairs of similar users [Node2Vec, LINE,...].

% \begin{figure}[h]
%     \centerline{\includegraphics[width=12.5cm]{figure/user_embedding}}
%     \caption{ user embedding } \label{user_embedding}
% \end{figure}


\begin{algorithm}[h]
\caption{User2Vec algorithm}
\label{alg:userembedding}
\begin{algorithmic}
\Function{User2Vec} {dimensions $d$, num\_samples $n$, item rating matrix $\bm{R}$, helpfulness rating tensor $\bm{H}$}
\State Initialize $clues$ to $empty$
\For {$iter = 1$ to $n$}
    \State append EnthusiastPair($\bm{R}$) to $clues$
    \State append SupporterPair($\bm{H}$) to $clues$
    \State append ReviewerSupporterPair($\bm{H}$) to $clues$
\EndFor
\State $userVec =$ Skip-Gram($clues, d$)
\State \Return $userVec$
\EndFunction

\Function{EnthusiastPair} {item rating matrix $\bm{R}$}
    \State let $Enthusiast(item)$ be $\{u | \bm{R}_{u,item}=ir_{max}\}$
    \State let $EI$ be $\{item | |Enthusiast(item)| \geq 2 \}$
    \State sample item $i$ from $EI$ with the prob. proportional to the cardinality of $Enthusiast(i)$
    \State sample user $u,v$ from $Enthusiast(i)$ uniformly at random
    \State \Return $(u,v)$
\EndFunction

\Function{SupporterPair} {review helpfulness rating tensor $\bm{H}$}
    \State let $Supporter(u,i)$ be $\{v | \bm{H}_{v,u,i}=hr_{max}\}$
    \State let $SU$ be $\{(u,i) | |Supporter(u,i)| \geq 2\}$
    \State sample review $(u,i)$ from $SU$ with the prob. proportional to the cardinality of $Supporter(u,i)$
    \State sample user $u_a,u_b$ from $Supporter(u,i)$ uniformly at random
    \State \Return $(u_a,u_b)$
\EndFunction

\Function{ReviewerSupporterPair} {review helpfulness rating tensor $\bm{H}$}
    \State let $Supporter(u,i)$ be $\{v | \bm{H}_{v,u,i}=hr_{max}\}$
    \State let $SU$ be $\{(u,i)| |Supporter(u,i)| \geq 1\}$
    \State sample review $(u,i)$ from $SU$ with the prob. proportional to the cardinality of $Supporter(u,i)$
    \State sample user $v$ from $Supporter(u,i)$ uniformly at random
    \State \Return $(u,v)$
\EndFunction

% \Function{CallA}{$a$} \funclabel{alg:a} \label{alg:a-line}
%     \State \Call{CalcSquare}{$a$}
% \EndFunction
% \Statex
% \Function{CalcSquare}{$b$} \funclabel{alg:b}
%     \State \Return $b\times b$
% \EndFunction

\end{algorithmic}
\end{algorithm}


\section{Robust Review Quality Measure}

We assume that the true quality of a review can be estimated by the mean of authentic helpfulness ratings.
However, in the presence of fake helpfulness ratings, we need to estimate the true quality of a review by the weighted mean of helpfulness ratings where the fake helpfulness ratings have very low weight.
If the amount of helpfulness ratings is sufficiently high, then this estimation get high confidecne.
However, for reviews that have few helpfulness ratings and reviews having only fake helpfulness ratings, the weighted mean is not robust estimator for such reviews.
Say a fake review which received only fake helpfulness ratings, then the weighted mean output is biased toward fake helpfulness ratings even if the weight of fake helpfulness ratings is almost zero.
With all of these things in mind, we define the following review quality measure to estimate the true quaity of a review.
\begin{equation}
Quality(u,i) = \frac{ w_{prior} Q_{default} + \sum_{v \in \{x|\bm{H}_{v,u,i} \neq ?\}} T(v,u) \times \bm{H}_{v,u,i} } {w_{prior}  + \sum_{v \in \{x|\bm{H}_{v,u,i} \neq ?\}} T(v,u) }
\end{equation}
We take Bayesian average approach [] that incorporates both a prior belief and a weighted mean of review helpfulness ratings associated with a review r.
The prior quality $Q_{default}$ works as prior belief. $w_{prior}$ is the weight given to the prior belief ($Q_{default}$).
In this work, we set $Q_{default}$ as the mean of helpfulness rating range (e.g. 2.5 in range from 0 to 5), and $w_{prior}$ as 1.
The weight of a review helpfulness rating in review quality estimation is determined by the following function $T: U \times U \rightarrow R$.
\begin{equation}
T(v,u)=
\begin{cases}
  exp(-\mu \times(cosine(userVec_v,userVec_u)-\theta)) & \text{if}\ cosine(userVec_v,userVec_u) \geq \theta \\
  1 & \text{otherwise}
\end{cases}
\end{equation}
% Armed with user2vec results, we define the suspiciousness of a helpfulness as a function of the similarity between the feature vectors of the helpfulness rater and the reviewer.
The function $T$ takes a rater and a reviewer, and output the trustfulness of the relationship between them.
$T$ penalizes the trustfulness if the cosine similarity between the two feature vectors of the reviewer and the helpfulness rater is larger than the threshold $\theta$.
As mentioned eariler, this policy might lower the weight of authentic helpfulness ratings, but the likelihood of making such a misjudgment is low in the moderately high dimensional feature spaces.
$\mu$ is a constant for amplification of similairty
In this work, we set the $\theta$ as 0.8 and the $\mu$ as 100

With the above mentioned robust measure, the quality of reviews with few helpfulness ratings will be close to the default quality $Q_{default}$,
while reviews with many helpfulness ratings given by the users whose similarity to the reviewer is not that high will have a quality score close to its average helpfulness rating.
Most importantly, reviews with many helpfulness ratings from the users similar to reviewer will have a helpfulness score close to the default quality $Q_{default}$.
In other words, our measure prevent fake helpfulness ratings from manipulating the quality of their fake review.







\begin{algorithm}[h]
\caption{Robust recommendation system}
\label{alg:RRS}
\begin{algorithmic}[]

\Function{RRS} {dimensions $d$, num\_samples $n$, item rating matrix $\bm{R}$, helpfulness rating tensor $\bm{H}$}
\State $userVec = $ User2Vec($d,n,\bm{R},\bm{H}$)
\For {all review $(u,i)$}
    \State $\bm{W}_{u,i} = Quality(u,i) = \frac{ w_{prior} Q_{default} + \sum_{v \in \{x|\bm{H}_{v,u,i} \neq ?\}} T(v,u) \times \bm{H}_{v,u,i} } {w_{prior}  + \sum_{v \in \{x|\bm{H}_{v,u,i} \neq ?\}} T(v,u) }$
\EndFor
\State Optimize $Cost(\bm{U},\bm{V} | \bm{W}, \bm{R})=\sum_{\bm{R}_{i,j} \neq ?} \bm{W}_{i,j}(  \bm{R}_{i,j} - (\bm{UV})_{i,j} )^2 + \lambda(||\bm{U}||_F^2+||\bm{V}||_F^2)$

\State \Return $\bm{U}$ and $\bm{V}$

\EndFunction

\end{algorithmic}
\end{algorithm}

% Most recommendation systems for unbiasedness do not allow users to rate the helpfulness of their own reviews, i.e. “self-rating” is prohibited.
% Prohibiting self-rating can be interpreted as the effect of self-rating on recommendation system should be zero.
% Since our goal is to prevent fake helpfulness ratings from manipulating the helpfulness of fake reviews, we relax the condition of self-rating to being a very high similarity between the feature vectors of reviewer and helpfulness rater.
% Instead of zeroing the effect of self-rating, we assign the effect of modified self-rating to a small value in reverse proportion to the similarity.
% The rationale behind this choice is that very high similarity between the feature vectors of reviewer and helpfulness rater might indicate group attack and mitigating the effect of suspicious helpfulness ratings leads to unbiased helpfulness measure.
% We define the following robust helpfulness measure which estimates the true helpfulness of a review by penalizing the modified “self-rating”.

% ====================================================================================================================
% ====================================================================================================================
% ====================================================================================================================
% ====================================================================================================================

\chapter{Experiment}
\section{Experimental Setting}
We use the publicly available dataset provided by \cite{ETAF}, namely CiaoDVD.
The CiaoDVD dataset contains users’ review and helpfulness rating information from ciao.dvd.co.uk where users rate DVDs and others’ reviews.
In the CiaoDVD dataset, users can rate items with a score from 1 to 5 as reviewer, and rate the helpfulness of reviews with a score from 0 to 5 as helpfulness rater.
From the original dataset, we filter out reviewer who rated less than five items and item that received less than five ratings.
The statistics of the resulting dataset are shown in Table \ref{tableCiaoDVD}.

\begin{table}[h]
\caption{Statistics of CiaoDVD dataset}
\label{tableCiaoDVD}
\begin{center}
\begin{tabular} {|c|c|}
\hline
\textbf{Features} & \textbf{CiaoDVD} \\ \hline
Reviewers & 1822 \\ \hline
Items & 2069 \\ \hline
Reviews & 28374 \\ \hline
Helpfulness Rater & 27900 \\ \hline
Helpfulness Rating & 661040 \\ \hline
\end{tabular}
\end{center}
\end{table}

We assume the original data is authentic.
To this data, we inject fake users, item ratings(reviews) and helpfulness ratings as mentioned in the Chapter 4.
Attack size, i.e. the number of injected fake users, ranges from 1\% to 3\% of total users.
Filler size is the number of $I^{filler}$ and popular size is the number of $I^{filler}$.
%Filler size + Selected size ~1\%.
We restrict the sum of filler size and popular size from exceeding 1\% of the total number of items.
We assume attacks with larger attack size and filler/popular size would be detected easily, so we decide to exclude them in our experiment setting.
% train test split
For performance evaluation, we perform 10-fold cross validation.
In each fold, the test set contains random 10\% original reviews, and the training set contains the remaining 90\% original reviews and all the fake reviews.
% target item / popular item filter,
We choose a set of target items $I^{target}$ as items which have been rated by at least 1\% users with below the median of the rating scale (3 in our rating scale [1,5]).
$I^{popular}$ consists of items rated items by at least 1\% users with the $ir_{max}$
Note that $I^{filler}$ is randomly selected for each fake user


\section{Metrics}
%Average helpfulness of reviews
\textbf{Average helpfulness of reviews} measures the average helpfulness of reviews belonging to each category.
We compare the average helpfulness values of fake reviews and authentic reviews to find out the robustness of a helpfulness measure.
A robust helpfulness measure would produce the small average helpfulness of fake reviews even in the presence of fake helpfulness ratings.

\begin{equation}
AverageHelpfulness(IR) = \frac{1}{N_{IR}} \sum_{ (u,i) \in IR } helpfulness(u,i)
\end{equation}
where $N_{IR}$ is the number of reviews in the set of review $IR$

%Prediction shift on the target items
\textbf{Prediction shift on the target items} measures the average of change in the prediction of genuine users for the attacked items before and after a shilling attack.
In other words, this metric measures the degree of success of an attack.
The smaller the value of this metric, the more robust the recommendation method is.
%//This metric is also sensitive to the strength of an attack, with stronger attacks causing larger prediction shift. [from attack resistant]
\begin{equation}
PredictionShift(U,V,U',V') = \frac{1}{|U^g||I^{target}|} \sum_{u \in U^g} \sum_{i \in I^{target}} (U'V')_{u,i}-(UV)_{u,i}
\end{equation}
where $(U'V')_{u,i}$ is the predicted rating value of user $u$ for item $i$ after an attack

\textbf{Mean Average Error(MAE) on test set} is the overall prediction error on ratings in the test set which contains 10\% of all item ratings of original users in the dataset.
MAE is commonly used to compare the predictive accuracy of recommendation algorithms.
In this paper we use MAE to measure the accuracy loss that is sacrificed to improve robustness.

\begin{equation}
MAE(U,V) = \frac{1}{|test set|} \sum_{R_{u,i} \in test set} |R_{u,i}-(UV)_{u,i}|
\end{equation}



\section{Results and Analysis}


We first visualize the results of User2Vec to show User2Vec's ability to capture fake users.
We use learned users' feature vectors as the input to the visualization toolt-SNE \cite{TSNE}.
The users are mapped to the 2-D space.
Diamond-shaped points represent fake user, while x-shaped green colored points represent normal users.
We observed that feature vectors of fake users are very close to each other.
\begin{figure}[h]
    \centerline{\includegraphics[width=10.5cm]{figure/user2vec_result}}
    \caption{  Visualization of the User2Vec result } \label{user2vec_result}
\end{figure}

% helpfulness measure comparison
% 1% attack size, 0.5 filler size, 0.5% popular size을 가지는 공격이 있고 차원의 크기는 32로 설정한 상황에서 우린 가짜 리뷰와 진실된 리뷰의 유용성의 평균값을 계산한다.

% we compute the average of the usefulness of fake and true reviews in the situation where fake data are injected by attacks with 1% attack size, 2% filler size, and 3% popular size. we set dimensions of user feature vectors to 32


% We first compare the average of helpfulness of fake and authentic reviews computed by each measure, varying attack size from 1\% to 3\% with fixed 0.5\% filler size, 0.5\% selected size and 32 dimensions of user feature vector.
We compute the average of the helpfulness of fake reviews and authentic reviews.
We inject fake review through attacks with 1\% attack size, 0.5\% filler size, and 0.5\% popular size.
We set dimensions of feature vectors to 32, which is a parameter needed in our helpfulness measure.
% result explanation
As shown in table ~\ref{resultHelpfulness}, the naive helpfulness measure results in fake reviews have a higher helpfulness than authentic reviews, whereas our helpfulness measure yields the opposite.
In specific, while the naive helpfulness measure computes the helpfulness of fake reviews at a value close to $hr_{max}$, our helpfulness measure computes the helpfulness of fake reviews at a value close to the default helpfulness $H_{default}$.
%We observed a slight decrease in the helpfulness of authentic reviews when using our method.
%The reason for this in
% even in the presence of an attack with 3\% attack size.

%As shown in table ~\ref{resultHelpfulness}, the results of the naive helpfulness measure indicate that fake reviews have a higher helpfulness than authentic reviews and increasing attack size promotes the helpfulness of fake reviews.
%While the naive helpfulness measure computes the helpfulness of fake reviews at a value close to $hr_{max}$, our helpfulness measure computes the helpfulness of fake reviews at a value close to the default helpfulness $H_{default}$ even in the presence of an attack with 3\% attack size.
%//Authentic reviews have similar helpfulness values as we expect


% Result (helpfulness)
%----------------------- naive attack True rank 30 lda 0.0001 ---------------------
%Helpfulness distribution
%[ Fake target] Mean 1.0 [10, 25, 50, 75, 90] 1.0 1.0 1.0 1.0 1.0
%[Honest  all ] Mean 0.707772781236 [10, 25, 50, 75, 90] 0.5 0.62 0.776 0.8 0.807407407407
%[Honest target] Mean 0.752601402364 [10, 25, 50, 75, 90] 0.613846153846 0.742721164614 0.79625 0.800427350427 0.812090322581
%----------------------- robust attack True rank 30 lda 0.0001 ---------------------
%Helpfulness distribution
%[ Fake target] Mean 0.500005215262 [10, 25, 50, 75, 90] 0.500000899783 0.500001278803 0.500001819832 0.500003403991 0.500012499609
%[Honest  all ] Mean 0.690197908439 [10, 25, 50, 75, 90] 0.5 0.59375 0.750000105113 0.788920774322 0.798
%[Honest target] Mean 0.742774662194 [10, 25, 50, 75, 90] 0.60599378882 0.7359375 0.783293269231 0.793191037632 0.803526694633

\begin{table}[h]
\caption{Review helpfulness results. The range of helpfulness is from 0 to 5}
\label{resultHelpfulness}
\begin{center}
\begin{tabular}{|c|c|c|c|c|}
\hline
\multirow{2}{*}{Attack Size} & \multicolumn{2}{c|}{\begin{tabular}[c]{@{}c@{}}Naïve Helpfulness Measure\end{tabular}} & \multicolumn{2}{c|}{Our Helpfulness Measure} \\ \cline{2-5}
                             & Fake Reviews                              & Authentic Reviews                              & Fake Reviews       & Authentic Reviews       \\ \hline
%1\%                          & 4.86                                      & 3.45                                           & 2.5                & 3.45                    \\ \hline
%3\%                          & 4.95                                      & 3.45                                           & 2.5                & 3.45                    \\ \hline
1\%                          & 5.0                                      & 3.5388                                           & 2.5                & 3.45                    \\ \hline
\end{tabular}
\end{center}
\end{table}



% Prediction shift
To compare the robustness, we compute the prediction shift on the target items of algorithms using different review helpfulness measures under attacks in variety of attack sizes, filler sizes, and popular sizes.
From Table ~\ref{resultPredictionShift}, we observe that our helpfulness measure leads to the lowest prediction shift on the target items in all conditions.
The reason for this is that fake item ratings have the least impact on prediction when applying our helpfulness measure rather than applying other methods.
Since fake helpfulness ratings increase the influence of fake item ratings on prediction when applying the naive helpfulness measure than when ignoring helpfulness, the naive helpfulness measure causes larger prediction shift on the target items than basic MF in the presence of fake helpfulness ratings.
Therefore we argue that our method resistant to review helpfulness manipulations adds robustness to WMF.


%Naive review helpfulness measure를 이용한 collaborative filtering은 helpfulness를 고려하지 않은 즉, 모든 review의 중요도를 동일하게 본 baseline method보다 latent feature가 fake review의 error를 낮추는 방향으로 최적화된다. 그로 인해 공격의 영향력이 더 강화되어 더 큰 prediction shift를 보이게 된다.

\begin{table}[h]
\caption{Prediction shift on the target items}
\label{resultPredictionShift}
\begin{center}
\begin{tabular}{|c|c|c|c|c|c|}
\hline
\textbf{Attack size} & \textbf{Filler Size} & \textbf{Popular Size} & \textbf{Base} & \textbf{Naive} & \textbf{Ours} \\ \hline
\multirow{3}{*}{1\%} & 1\%                  & 0\%                   & 1.01465       & 1.7805         & \textbf{0.092686} \\ \cline{2-6}
                     & 0.50\%               & 0.50\%                & 1.48154       & 1.90081        & \textbf{0.239489} \\ \cline{2-6}
                     & 0\%                  & 1\%                   & 1.72337       & 1.79765        & \textbf{0.246163} \\ \hline
\multirow{3}{*}{2\%} & 1\%                  & 0\%                   & 1.5178        & 2.28872        & \textbf{0.172815} \\ \cline{2-6}
                     & 0.50\%               & 0.50\%                & 1.91454       & 2.30772        & \textbf{0.387328} \\ \cline{2-6}
                     & 0\%                  & 1\%                   & 2.08557       & 2.1092         & \textbf{0.469354} \\ \hline
\multirow{3}{*}{3\%} & 1\%                  & 0\%                   & 1.79902       & 2.5618         & \textbf{0.307992} \\ \cline{2-6}
                     & 0.50\%               & 0.50\%                & 1.89458       & 2.49592        & \textbf{0.580003} \\ \cline{2-6}
                     & 0\%                  & 1\%                   & 1.99723       & 2.24161        & \textbf{0.659305} \\ \hline
\end{tabular}
\end{center}
\end{table}

% MAE result (ver 1.0)
We also investigated the predictive performance of relying on each review helpfulness measure.
We compute MAE on the test set when the training set contains fake item ratings.
% result
According to ~\ref{resultMAE}, basic matrix factorization performs better than others.
% discussion
Since MAE gives equal importance to the errors of all the ratings in the test set, it is not ideal metric to measure the performance of weighted matrix factorizations which allow unimportant ratings to have significant errors.
% While weighted matrix factorization tends to be fitted more toward important ratings, MAE gives equal importance to all ratings of the test set.
Even though MAE is not proper metric for weighted matrix factorization, the performances are not significantly different from basic matrix factorization.
We compute Cohen's $d$ for the effect size based on means of predictive errors between basic MF and weighted MF using our helpfulness measure.
The value of $d$ is near 0.02, which is a small value according to \cite{EffectSize}.
Therefore, weighted matrix factorization using our helpfulness measure provides robustness at a not significant additional cost of predictive accuracy.


%추천 시스템의 본연의 목적은 사용자의 예상 평점을 맞추는 것이기 때문에 예상 평점을 맞추는 퍼포먼스를 비교한다. 공격에 의해 아무리 바뀌지 않는 방법이더라도 예상 평점의 정확도가 너무 떨어지는 방법은 바람직하지 않다.
%//예를 들면 overall mean rating(4.02314289071)로 전부 예상하면 MAE는 0.820751825687 (더 좋다….). 우리의 방법은 attack이 1\%일땐 naïve가 ours보다 좋은데 3\%일땐 그 반대가 된다. 이유는…..몰라
%performance에서 손해를 보지만 그 손해가 몇 퍼센트 차이가 나지 않는다.
%이 metric은 test dataset에 소속된 rating의 중요도를 모두 1로 보았기 때문에 review helpfulness를 고려한 추천 시스템의 퍼포먼스를 측정하기엔 완벽한 선택은 아니다. 그럼에도 불구하고 base와 그리 차이 나지 않는 performance를 보여주기 때문에 우리 방법도 괜찮다(?)


\begin{table}[h]
\caption{MAE on test set}
\label{resultMAE}
\begin{center}
\begin{tabular}{|c|c|c|c|c|c|}
\hline
\textbf{Attack Size} & \textbf{Filler Size} & \textbf{Popular Size} & \textbf{Base} & \textbf{Naive} & \textbf{Ours}     \\ \hline
\multirow{3}{*}{1\%} & 1\%                  & 0\%                   & 0.821944      & 0.831673       & \textbf{0.843252} \\ \cline{2-6}
                     & 0.5\%               & 0.5\%                & 0.825909      & 0.836473       & \textbf{0.842159} \\ \cline{2-6}
                     & 0\%                  & 1\%                   & 0.825078      & 0.837706       & \textbf{0.84074}  \\ \hline
\multirow{3}{*}{2\%} & 1\%                  & 0\%                   & 0.820305      & 0.826343       & \textbf{0.840074} \\ \cline{2-6}
                     & 0.5\%               & 0.5\%                & 0.826043      & 0.834851       & \textbf{0.836148} \\ \cline{2-6}
                     & 0\%                  & 1\%                   & 0.822775      & 0.847235       & \textbf{0.843203} \\ \hline
\multirow{3}{*}{3\%} & 1\%                  & 0\%                   & 0.816267      & 0.828714       & \textbf{0.840653} \\ \cline{2-6}
                     & 0.5\%               & 0.5\%                & 0.825463      & 0.846773       & \textbf{0.841859} \\ \cline{2-6}
                     & 0\%                  & 1\%                   & 0.829685      & 0.85275        & \textbf{0.841454} \\ \hline
\end{tabular}
\end{center}
\end{table}


\chapter{Conclusion}
This paper proposes a robust review helpfulness measure so that collaborative filtering using review helpfulness information successfully restricts the effect of shilling attacks. %fake reviews.
%Specifically, given an item rating matrix and a review helpfulness rating tensor, we map users in recommendation system to feature vectors which encode behavior related with group attack.
Specifically, given an item rating matrix and a review helpfulness rating tensor, we learn representations of users in recommendation system.
These representations encode behaviors associated with shilling attack.
% not sure "estimate A by B"
Armed with user representations, we estimate the true helpfulness of reviews by the Bayesian weighted average of review helpfulness ratings.
We penalize the weight of a helpfulness rating if the helpfulness rater and reviewer are suspected of having a relationship of self-rating.
Experimental results on a real-world dataset demonstrate the robustness of our method against review helpfulness manipulation.
Future research directions include developing robust measures against nuke attacks and more elaborate attacks.

%%
%% 참고문헌 시작
%% bibliography
%% It can be changed but should include sufficient information.
\begin{thebibliography}{00}


%\bibitem{ML2} I. Song, T. An, and J. Oh, \textit{Near ML decoding method based on metric-first search and branch length threshold,} registration no. US 8018828 B2, Sep. 13, 2011, USA.
%\bibitem{SOCA2} H.-K. Min, T. An, S. Lee, and I. Song, “Non-intrusive appliance load monitoring with feature extraction from higher order moments,” in \textit{Proc. 6th IEEE Int. Conf. Service Oriented Computing, Appl.,} Kauai, HI, USA, pp. 348-350, Dec. 2013.
%\bibitem{EF2} I. Song and S. Lee, “Explicit formulae for product moments of multivariate Gaussian random variables,” \textit{Statistics, Probability Lett.,} vol. 100, pp. 27-34, May 2015.
\bibitem{yelp_study} M. Luca. Reviews, Reputation, and Revenue: The Case of Yelp.com. Harvard Business School Working Papers,2011. http://www.hbs.edu/research/pdf/12-016.pdf

\bibitem{MF_CF} Koren, Yehuda, Robert Bell, and Chris Volinsky. "Matrix factorization techniques for recommender systems." Computer 42.8 (2009): 30-37.

\bibitem{RMF} B. Mehta, T. Hofmann, and W. Nejdl, “Robust Collaborative Filtering,” RecSys, p. 49, 2007.

\bibitem{AttackResistant} B. Mehta and W. Nejdl, “Attack resistant collaborative filtering,” Proc. 31st Annu. Int. ACM SIGIR Conf. Res. Dev. Inf. Retr. SIGIR 08, p. 75, 2008.

%review text
\bibitem{naive_helpfulness} S. Kim, P. Pantel, T. Chklovski, and M. Pennacchiotti. Automatically assessing review helpfulness. EMNLP, 2006

%review text
\bibitem{text_feature_1} Liu, Jingjing, et al. "Low-Quality Product Review Detection in Opinion Summarization." EMNLP-CoNLL. Vol. 7. 2007.

%review text
\bibitem{text_feature_2} Ott, Myle, et al. "Finding deceptive opinion spam by any stretch of the imagination." Proceedings of the 49th Annual Meeting of the Association for Computational Linguistics: Human Language Technologies-Volume 1. Association for Computational Linguistics, 2011.

\bibitem{RQMF} S. Raghavan, S. Gunasekar, and J. Ghosh. Review quality aware collaborative filtering. In Proceedings of the sixth ACM conference on Recommender systems, pages 123–130. ACM, 2012.

\bibitem{DualRole} S. Wang, J. Tang, and H. Liu, “Toward Dual Roles of Users in Recommender Systems,” Cikm, pp. 1651–1660, 2015.

\bibitem{LiesAndPropaganda} Mehta, Bhaskar, Thomas Hofmann, and Peter Fankhauser. "Lies and propaganda: detecting spam users in collaborative filtering." Proceedings of the 12th international conference on Intelligent user interfaces. ACM, 2007.


\bibitem{UnsupervisedShilling} Mehta, Bhaskar. "Unsupervised shilling detection for collaborative filtering." AAAI. 2007.

\bibitem{DeepWalk} B. Perozzi, R. Al-Rfou, and S. Skiena. DeepWalk: Online learning of social representations. In KDD, 2014.

\bibitem{Word2Vec} T. Mikolov, K. Chen, G. Corrado, and J. Dean. Efficient estimation of word representations in vector space. In ICLR, 2013.

\bibitem{NegativeSampling} T. Mikolov, I. Sutskever, K. Chen, G. S. Corrado, and J. Dean. Distributed representations of words and phrases and their compositionality. In NIPS, 2013.

\bibitem{Node2Vec} Grover Aditya and  Leskovec Jure. ”node2vec: Scalable Feature Learning for Networks.” Proceedings of the 22nd ACM SIGKDD International Conference on Knowledge Discovery and Data Mining. ACM, 2016.

\bibitem{ETAF} Guo, Guibing, et al. "Etaf: An extended trust antecedents framework for trust prediction." Advances in Social Networks Analysis and Mining (ASONAM), 2014 IEEE/ACM International Conference on. IEEE, 2014.

\bibitem{ImplicitCF} Hu, Yifan, Yehuda Koren, and Chris Volinsky. "Collaborative filtering for implicit feedback datasets." 2008 Eighth IEEE International Conference on Data Mining. Ieee, 2008.

\bibitem{EffectSize}  Cohen, Jacob (1988). Statistical Power Analysis for the Behavioral Sciences. Routledge. ISBN 1-134-74270-3.

\bibitem{TNSE} L. Van der Maaten and G. Hinton. Visualizing data using t-sne. Journal of Machine Learning Research, 9(2579-2605):85, 2008

\end{thebibliography}





%%
%% 감사의 글 시작
%% Acknowledgement
%%
% @command acknowledgement 감사의글
% @options [1 | 2 | 3 |4 ]
% - 1 : 본문과 감사의 글이 둘 다 한글일 때  | 2 : 본문은 한글인데 감사의 글이 영어일 때 | 3 :  본문과 감사의 글이 둘 다 영어일 때  | 4 : 본문은 영어인데 감사의 글이 % 한글일 때
%% It is optional.

\acknowledgment[4]
    감사합니다.

%%
%% 약력 시작
%% Curriculum Vitae
%%
% @command curriculumvitae 이력서
% @options [1 | 2 | 3 |4 ]
% - 1 : 본문과 약력이 둘 다 한글일 때  | 2 : 본문은 한글인데 약력이 영어일 때 | 3 :  본문과 약력이 둘 다 영어일 때  | 4 : 본문은 영어인데 약력이 한글일 때
%% It is optional and you can change form of this in the class file if you want.
\curriculumvitae[4]

    % @environment personaldata 개인정보
    % @command     name         이름
    %              dateofbirth  생년월일
    %              birthplace   출생지
    %              domicile     본적지
    %              address      주소지
    %              email        E-mail 주소
    % - 위 6개의 기본 필드 중에 이력서에 적고 싶은 정보를 입력
    % input data only you want
    \begin{personaldata}
        \name       {심 동 진}
        \dateofbirth{1992}{8}{4}
        \birthplace {경기도 송탄시}
        \email      {djsim@kaist.ac.kr}
    \end{personaldata}

    % @environment education 학력
    % @options [default: (none)] - 수학기간을 입력
    \begin{education}
        \item[2008. 3.\ --\ 2011. 2.] 거창고등학교
        \item[2011. 3.\ --\ 2014. 8.] 성균관대학교 소프트웨어학과 (학사)
        \item[2015. 3.\ --\ 2017. 2.] 한국과학기술원 전산학부 (석사)
    \end{education}

    % @environment career 경력
    % @options [default: (none)] - 해당기간을 입력
    \begin{career}
        \item[2015. 9.\ --\ 2016. 8.] 한국과학기술원 전산학부 조교
    \end{career}

    % @environment activity 학회활동
    % @options [default: (none)] - 활동내용을 입력
%%    \begin{activity}
%%        \item J. Choi, \textbf{Yong-Hyun Kim}, K.J. Chang, and D. Tomanek,
%%             \textit{Occurrence of itinerant ferromagnetism in C/BN superlattice
%%             nanotubes}, 5th Asian Workshop on First-Principles Electronic
%%             Structure Calculations, Seoul (Korea), October., 2002.
%%    \end{activity}

    % @environment publication 연구업적
    % @options [default: (none)] - 출판내용을 입력
%     \begin{publication}
%         %\item \textbf{Yong-Hyun Kim}, J. Choi, K.J. Chang, and D. Tomanek,
%         %     \textit{Magnetic instability in partly opened C$_{60}$ isomers},
%         %     in preparation.
%         \item H.-K. Min, Y. Hou, {\bf S. Park}, and I. Song,
% ``A computationally efficient scheme for feature extraction with kernel discriminant analysis,"
% \textit{Patt. Recogn.}, vol.~50, no.~2, pp.~45-55, Feb. 2016 (to be published).
%     \end{publication}

%% 본문 끝
\label{paperlastpagelabel} %마지막 페이지 위치 지정
\end{document}
